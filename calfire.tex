% Created 2016-12-30 Fri 16:26
\documentclass[a4paper]{article}
\usepackage[utf8]{inputenc}
\usepackage[T1]{fontenc}
\usepackage{fixltx2e}
\usepackage{graphicx}
\usepackage{longtable}
\usepackage{float}
\usepackage{wrapfig}
\usepackage{rotating}
\usepackage[normalem]{ulem}
\usepackage{amsmath}
\usepackage{textcomp}
\usepackage{marvosym}
\usepackage{wasysym}
\usepackage{amssymb}
\usepackage{hyperref}
\tolerance=1000
\usepackage{amssymb,amsmath}
\usepackage{natbib}
\usepackage[margin=2cm]{geometry}
\usepackage{fancyhdr} %For headers and footers
\pagestyle{fancy} %For headers and footers
\usepackage{acronym}
\usepackage{lastpage} %For getting page x of y
\usepackage{float} %Allows the figures to be positioned and formatted nicely
\floatstyle{boxed} %using this
\usepackage{draftwatermark}
\restylefloat{figure} %and this command
\usepackage{url} %Formatting of yrls
\rhead{\includegraphics[width=3cm]{berkeley}}
\chead{}
\lfoot{Draft}
\cfoot{}
\setlength{\parskip}{1em}
\rfoot{\thepage\ of \pageref{LastPage}}
\acrodef{GHG}{Greenhouse Gas}
\acrodef{SLCP}{Short-Lived Climate Pollutants}
\acrodef{CAP}{Criteria Air Pollutants}
\acrodef{PM}{Particulate Matter}
\acrodef{NOX}{Oxides of Nitrogen}
\acrodef{CO2e}{Carbon Dioxide Equivalents}
\acrodef{CARB}{California Air Resources Board}
\acrodef{DF}{Displacement Factor}
\acrodef{FCP}{Forest Climate Plan}
\acrodef{BOF}{California Board of Forestry}
\acrodef{BC}{Black Carbon}
\acrodef{TC}{Total Carbon}
\acrodef{BOE}{California Board of Equalization}
\acrodef{TPO}{Timber Products Output}
\author{Peter Tittmann, Ph.D.}
\date{\today}
\title{Estimating Total Emissions from Biomass Burning}
\hypersetup{
  pdfkeywords={},
  pdfsubject={},
  pdfcreator={Emacs 24.4.1 (Org mode 8.2.10)}}
\begin{document}

\maketitle
\tableofcontents

To arrive at an estimate of total CO2 equivalent (CO2e) emissions for 2015, 
we combine \ac{BC} emissions estimates from the \ac{CARB}
Criteria Air Pollutant Emissions Inventory with GHG emissions estimates reported in the
CARB GHG Emissions Inventory and the CONSUME model. The time discrepancy between the most recent GHG Emissions Inventory in 2004 and the current year is
acknowledged as an irreconcilable source of uncertainty in this
estimation. The use of the COMSUME moel to predict pile buring emissions is likely more reliable and as pile buring is a central focus of this analysis we confidently use the emissions estimated here for pile burning thourghout. Model-based estimation could be used to derive a ratio of \ac{GHG} to PM using the \href{http://www.fs.fed.us/pnw/fera/research/smoke/consume/index.shtml}{USFS CONSUME} model. Overall, this analysis demonstrates that
substantial emissions from forest management residuals have been reported by CARB emissions inventories and that such inventories may be utilized to establish a baseline condition. We find that a rough estimate of CO2e emissions from pile
burning annual approaches 1 Mt CO2e Table \ref{pile_summary}.

\begin{center}
\begin{tabular}{rl}
Mt CO2e & Source\\
\hline
1.4071705 & CO2e pile burning\\
1.1073965 & CO2e BC pile burning\\
\hline
2.514567 & Total Mt CO2e\\
\end{tabular}
\end{center}
\emph{\emph{BC emissions in terms of CO2e has not been included in any \ac{GHG} emissions
inventory published by CARB. Remove?}}
% Emacs 24.4.1 (Org mode 8.2.10)
\end{document}
