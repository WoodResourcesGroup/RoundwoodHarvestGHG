% Created 2016-12-30 Fri 10:02
\documentclass[a4paper]{article}
\usepackage[utf8]{inputenc}
\usepackage[T1]{fontenc}
\usepackage{fixltx2e}
\usepackage{graphicx}
\usepackage{longtable}
\usepackage{float}
\usepackage{wrapfig}
\usepackage{rotating}
\usepackage[normalem]{ulem}
\usepackage{amsmath}
\usepackage{textcomp}
\usepackage{marvosym}
\usepackage{wasysym}
\usepackage{amssymb}
\usepackage{hyperref}
\tolerance=1000
\usepackage{amssymb,amsmath}
\usepackage{natbib}
\usepackage[margin=2cm]{geometry}
\usepackage{fancyhdr} %For headers and footers
\pagestyle{fancy} %For headers and footers
\usepackage{lastpage} %For getting page x of y
\usepackage{float} %Allows the figures to be positioned and formatted nicely
\floatstyle{boxed} %using this
\usepackage{draftwatermark}
\restylefloat{figure} %and this command
\usepackage{url} %Formatting of yrls
\rhead{\includegraphics[width=3cm]{berkeley}}
\chead{}
\lfoot{Draft}
\cfoot{}
\setlength{\parskip}{1em}
\rfoot{\thepage\ of \pageref{LastPage}}
\author{Peter Tittmann, Ph.D.}
\date{\today}
\title{Emissions reductions from harvested wood products and management residuals}
\hypersetup{
  pdfkeywords={},
  pdfsubject={},
  pdfcreator={Emacs 24.4.1 (Org mode 8.2.10)}}
\begin{document}

\maketitle
\tableofcontents

\pagebreak
\section{California Forest Management Emissions Profile}
\label{sec-1}

Forest management activities in California produce logs for lumber markets while maintaining and enhancing forest health. In addition to merchantable logs, these harvest activities produce logging residuals that are either left in the stand to decompose, or pile burned as directed by forest practice rules (\href{http://calfire.ca.gov/resource_mgt/downloads/2013_FP_Rulebook_with_Tech_RuleNo1.pdf}{California Forest Practice Rules}, Article 7 §
917.2). By utilizing residual wood biomass produced from these forest management activities, there exists potential to reduce emissions from greenhouse gas (GHG) and other climate pollutants.

The majority of biomass produced from forest management activities end up as residual biomass material that is either left in the woods to decompose, or aggregated at a landing where it is eventually burned. When considered alongside the accumulation of woody material from historic fire suppression activity, there exists a heightened volume of combustible woody material in excess of historic reference conditions, elevating the risk of damaging wildfire in much of California’s forestland. Currently, common practice for fuel load management in California forests involve prescribed natural fire and sanitation pile burning. However, these approaches have limitatons as previous studies have demonstrated that prescribed natural fire is often only effective at reducing fuel loading and maintaining fire resilient landscapes when it is coupled with mechanical treatment to remove biomass (Stephens et al 2009). Open pile burning of residual biomass can also result in higher emissions of strong radiative forcing agents (black carbon) and criteria air pollutants (PM, NOX) when compared to controlled combustion in biomass power plants with modern emissions control technology. 

Ultimately, any combustion or decomposition of residual material results in emissions of greenhouse gases (GHG), criteria air pollutants (CAP), and short-lived climate pollutants (SLCP). Even in the absence of forest management activity, atmospheric emissions are produced from 
stochastic processes such as wildfire, pest, and disease outbreaks. As such, utilization strategies are necessary to reduce the air quality impact of common forestry practice. Furthermore, the opportunity cost of residual biomass use in bioenergy and product applications weigh in favor of diverting residual materral streams towards alternative utilization strategies. Figure \ref{fig:wood_fates} presents an overview of emissions and emissions reduction pathways for wood from California's forests. 

\begin{figure}[htb]
\centering
\includegraphics[width=0.75\textwidth]{./graphics/wood_fates_rs.pdf}
\caption{Overview of fates of wood resulting from harvest and mortality in California forests. Note that time is not represented in this figure. \label{fig:wood_fates}}
\end{figure}


The focus of this analysis is on deriving emissions associated with \textbf{management activity}. This report does not assess greenhouse gas emissions from pest or disease induced mortality, which is estimated at approximately 34 MMT of CO2 equivalent(CO2e) annually in California forests \cite{Christensen2016}. Emissions from mortality are indirectly related to management activities just as wildfire is, and must be accounted for to comprehensively evaluate the climate impacts of harvesting. To estimate emissions affected by forest  management activity, it is necessary to take the following steps:

\begin{enumerate}
\item Estimate CO2 equivalent (CO2e) emissions from burning forest management
residuals using Criteria Air Pollutant (CAP) and Greenhouse Gas (GHG) emissions inventories
published by the California Air Resources Board (CARB)

\item Estimate the volume and fate of wood that is removed, left in the
forest, and burned as a result of direct anthropogenic management
activities.

\item Establish life-cycle displacement factors (DF) for all
utilized wood and apply DF to harvested wood to obtain an aggregate estimate.
\end{enumerate}

\subsection{Report Objectives}
\label{sec-1-1}

Quantifying the climate effects of wood products and forest management
residuals is important to the development of the Forest Climate Plan
(FCP)\footnote{The \href{http://www.fire.ca.gov/fcat/}{Forest Climate Action Team} (FCAT) was assembled in August of 2014 with the primary purpose of developing a Forest Carbon Plan by the end of 2016. FCAT is comprised of Executive level members from many of the State’s natural resources agencies, state and federal forest land managers, and other key partners directly or indirectly involved in California forestry. FCAT is under the leadership of CAL FIRE, Cal-EPA, and The Natural Resources Agency.} as well as efforts underway by the California Board of
Forestry and CalFire to meet the intent of AB 1504 (2010)\footnote{\href{http://leginfo.legislature.ca.gov/faces/billTextClient.xhtml?bill_id=200920100AB1504}{AB-1504} Forest resources: carbon sequestration.(2009-2010)}. To
inform these efforts, this report provides estimates of the following:

\begin{itemize}
\item GHG and SLCP emissions produced from the combustion or
decomposition of logging residuals.
\item GHG emissions reductions from the use of wood products harvested in
the state.
\end{itemize}


Estimates are based empirical data and reflect past forest
management activities. It is \textbf{critical} to note that the empirical
data used in this analysis reflect point-in-time measures that are
affected by a dynamic system of climate, growth, and mortality as well as macroeconomic and policy forces. This analysis may provide insight into
opportunities to more effectively utilize woody biomass residuals from
current forest management activities to reduce emissions. 

\subsection{Key Findings}
\label{sec-1-2}
\begin{itemize}
\item Baseline GHG and SLCP emissions from burning of forest
management residuals can be estimated and should be considered in
any forest management emissions baseline.

\item Total emissions from pile burning of forest management residuals
(including SLCP and GHG components) extrapolated from CARB emissions
inventory are estimated at \textbf{1.16 MTCO2e}

\item Wood harvested in California in 2012 resulted in avoided emissions of
\textbf{2.29 MMTCO2e}

\item Logging residuals not used in bioenergy production contributed annual
emissions of:
\begin{itemize}
\item \textbf{XXX MMTCO2e} resulting from anthropogenic burning of logging residuals

\item \textbf{XXX MMTCO2e} resulting from decomposition of logging residuals left
unburned
\end{itemize}

\item Un-utilized slash from non-commercial management activities on
National Forest System lands contributed emissions of XXX MMTCO2e

\item Forest Inventory and Analysis re-sample data has been used in the
southeast to quantify removals resulting from non-commercial
management activity and could be used for this purpose in California

\item The \href{https://ssl.arb.ca.gov/pfirs/}{Prescribed Fire Information Reporting System} (PFIRS) may be a useful tool for quantifying
emissions from pile burns and prescribed fire. It is a requirement that prescribed fires and pile
burns on National Forest System Lands are reported through PFIRS. However, California Air Quality Management
Districts are not required to report emissions through this system at this time. Therefore, it is not possible to associate burns in the PFIRS with commercial harvest activities.
\end{itemize}

\section{Estimating CO2 Equivalent Emissions from In-Forest Biomass Combustion}
\label{sec-2}


The California Air Resources Board (CARB) reports on
emissions from in-forest biomass combustion with current Greenhouse Gas (GHG) and Criteria Air
Pollutant (CAP) 
\href{http://www.arb.ca.gov/ei/ei.htm}{emissions inventories}. Both are necessary resources for establishing
aggregate annual climate-forcing emissions (Figure \ref{fig:burn_diag}). The GHG inventory captures
gasses with radiative forcing properties including CO2 and CH4, but does not capture elemental
carbon or black carbon (BC) emissions which also have strong radiative
forcing properties (Table \ref{tab:bc_gwp}). The \citet{CaliforniaAirResourcesBoard2015,CaliforniaAirResourcesBoard2016}
CAP report captures SLCP emissions from wildfire
(\texttt{80.52} MMTCO2e) and prescribed fire
(\texttt{3.66} MMTCO2e) from which black carbon emissions may be estimated. However, no reference in the CAP report is made to the source of these
SLCP estimates. When viewed in aggregate, a comprehensive reporting of total climate impact from anthropogenic burning may be estimated. 


\begin{figure}[htb]
\centering
\includegraphics[width=0.75\textwidth]{./graphics/burning.pdf}
\caption{Data sources available from CARB for estimating GHG and SLCP emissions from forest management. \label{fig:burn_diag}}
\end{figure}

\begin{itemize}
\item The GHG inventory captures
\end{itemize}
gasses with radiative forcing properties including CO2 and CH4, but does
not capture other emissions with strong radiative forcing properties such as elemental
carbon and black carbon (BC) (Table \ref{tab:bc_gwp}). 

\begin{table}[htb]
\caption{Range of Global Warming Potential(GWP) values for Black Carbon.\label{tab:bc_gwp}}
\centering
\begin{tabular}{rrrrrrl}
GWP$_{\text{20}}$ & GWP$\sigma$$_{\text{20}}$ & GWP$_{\text{100}}$ & GWP$\sigma$$_{\text{100}}$ & GWP$_{\text{500}}$ & GWP$\sigma$$_{\text{500}}$ & Source\\
\hline
2200.0 & 888.82 & 633.33 & 255.41 & 193.33 & 77.67 & Fuglestvedt2010\\
3200.0 &  & 900.0 &  &  &  & CaliforniaAirResourcesBoard2015\\
\end{tabular}
\end{table}


\begin{itemize}
\item The \citet{CaliforniaAirResourcesBoard2015,CaliforniaAirResourcesBoard2016}
\end{itemize}
CAP report captures SLCP emissions from wildfire
(\texttt{80.52} MMTCO2e) and prescribed fire
(\texttt{3.66} MMTCO2e) from which black carbon emissions may be estimated. However, no reference in the CAP report is made to the source of these
SLCP estimates. 

When viewed in aggregate, a comprehensive reporting of total climate impact from anthropogenic burning may be estimated. 

\subsection{Estimating Black Carbon Emissions from Biomass Burning}
\label{sec-2-1}

Black carbon (BC) is not directly reported by statewide emissions summaries. BC is a fraction of the Total 
Carbon (TC) component measured in particulate matter (PM 2.5) emissions published annually by the CARB \href{http://www.arb.ca.gov/ei/emissiondata.htm}{Criteria air pollutant (CAP)
emissions estimates}. By using the 2015 CAP emissions estimates shown in Table \ref{arb_pm_ann} with estimated ratios of 
smoldering to flaming combustion for hand/machine piled burns, prescribed 
natural fire and wildfire from \citet{Ward1989}, Black Carbon emissions
can be calculated from PM
2.5 with the following equations Eq. \eqref{eq-bc} :


\begin{center}
\begin{tabular}{lr}
Source & PM 2.5 (t y$^{\text{-1}}$)\\
\hline
ALL VEGETATION & 137630.15\\
FOREST MANAGEMENT & 5480.51\\
WILDLAND FIRE USE (WFU) & 6802.43\\
\end{tabular}
\end{center}

\begin{table}[htb]
\caption{Emissions of PM 2.5 in 2015 as reported by CARB \label{tab:arb_pm_ann}}
\centering
\begin{tabular}{lr}
Source & PM 2.5 (t)\\
\hline
ALL VEGETATION & 137630.15\\
FOREST MANAGEMENT & 5480.51\\
WILDLAND FIRE USE (WFU) & 6802.43\\
\end{tabular}
\end{table}



\begin{align}
BC &= \left( PM_{2.5} \times F \times TC_f \times BC_f\right) + \left( PM_{2.5} \times S \times TC_s \times BC_s\right) \label{eq-bc} \\
\text{where:} \nonumber \\
BC &= \text{Black Carbon (mass units)} \nonumber \\
PM_{2.5} &= PM_{2.5} \text{ (mass units)} \nonumber \\
F &= \text{Percent of combustion in flaming phase} \nonumber \\
TC_f &= \text{Total Carbon fraction of } PM_{2.5} \text{ for flaming phase} \nonumber \\
BC_f &= \text{Black Carbon fraction of Total Carbon for flaming phase} \nonumber \\
S &= \text{Percent of combustion in smoldering phase} \nonumber \\
TC_s &= \text{Total Carbon fraction of } PM_{2.5} \text{ for smoldering phase} \nonumber \\
BC_s &= \text{Black Carbon fraction of Total Carbon for smoldering phase} \nonumber
\end{align}


The variable smoldering to flaming combustion behavior of each biomass burning scenario produces different
ratios of BC emissions. To arrive at a rough estimate of BC emissions based on PM2.5, the
following steps are taken:

\begin{enumerate}
\item Determine the amount of PM2.5 produced in the flaming and smoldering
phases of combustion for each type (piles, prescribed,
wildfire). Ratios from \citet{Ward1989} and \citet{Jenk1996} (Table \ref{tab:bc_pm}) are used.
\end{enumerate}

\begin{table}[htb]
\caption{Factors used for calculating Black Carbon (BC) emissions from the three primary combustion sources. BC is a fraction of Total Carbon (TC) which is a fraction of total PM 2.5. Coefficients of variation (C$_{\text{v}}$) are reported here as well. \label{tab:bc_pm}}
\centering
\begin{tabular}{lrrrrrr}
Source & BC$_{\text{f}}$ t$^{\text{-1}}$ PM & TC$_{\text{f}}^{\text{Cv}}$ t$^{\text{-1}}$ PM & BC$_{\text{f}}^{\text{Cv}}$ t$^{\text{-1}}$ TC & BC$_{\text{s}}$ t$^{\text{-1}}$ PM 2.5 & TC$_{\text{s}}^{\text{Cv}}$ t$^{\text{-1}}$ PM & BC$_{\text{s}}^{\text{Cv}}$ t$^{\text{-1}}$ TC\\
\hline
Pile Burn & 0.046904 & 0.09 & 0.45 & 0.01624 & 0.01 & 0.49\\
Prescribed & 0.08016309 & 0.0733 & 0.5833 & 0.020944 & 0.08 & 0.29\\
Wildfire & 0.05870124 & 0.0867 & 0.4467 & 0.0228641 & 0.06 & 0.338\\
\end{tabular}
\end{table}

\begin{center}
\begin{tabular}{lrrrrrr}
Source & BC$_{\text{f}}$ t$^{\text{-1}}$ PM & TC$_{\text{f}}^{\text{Cv}}$ t$^{\text{-1}}$ PM & BC$_{\text{f}}^{\text{Cv}}$ t$^{\text{-1}}$ TC & BC$_{\text{s}}$ t$^{\text{-1}}$ PM 2.5 & TC$_{\text{s}}^{\text{Cv}}$ t$^{\text{-1}}$ PM & BC$_{\text{s}}^{\text{Cv}}$ t$^{\text{-1}}$ TC\\
\hline
Pile Burn & 0.046904 & 0.09 & 0.45 & 0.01624 & 0.01 & 0.49\\
Prescribed & 0.08016309 & 0.0733 & 0.5833 & 0.020944 & 0.08 & 0.29\\
Wildfire & 0.05870124 & 0.0867 & 0.4467 & 0.0228641 & 0.06 & 0.338\\
\end{tabular}
\end{center}


\begin{enumerate}
\item Define 1000 normal probability distributions using the coefficient
of variation from Table \ref{tab:bc_pm} to determine the percent of PM 2.5
comprised by carbonaceous material (TC), and the percent of TC comprised
by black carbon (BC). \emph{\emph{???Give estimates and coefficient of variation
estimates provided by \citet{Ward1989}, tables 2 and 3.???}}

\item Estimate annual BC emissions \ref{tab:carb_bc} based on probability distributions
defined in step 2 and annual emissions provided by CARB.
\end{enumerate}


\begin{table}[htb]
\caption{Annual black carbon emissions calculated from CARB volumes \label{tab:carb_bc}}
\centering
\begin{tabular}{lrrr}
Source & PM 2.5 (t y$^{\text{-1}}$) & BC (t y$^{\text{-1}}$) & GWP (t y$^{\text{-1}}$)\\
\hline
ALL VEGETATION & 137630.15 & 11225.85 & 35922719.54\\
FOREST MANAGEMENT & 5480.51 & 346.06 & 1107396.54\\
WILDLAND FIRE USE (WFU) & 6802.43 & 687.77 & 2200877.13\\
\end{tabular}
\end{table}



Given the variance in baseline assumptions (maybe elaborate? FOR PETER)of
BC volume in Table \ref{tab:bc_pm}, it is critical to acknowledge the 
minimum and maximum range of probable emissions volumes.
The following plot represents estimates of total BC emissions resulting
from combustion of woody biomass in wildfire, pile burning, and
prescribed natural fire based on CARB CAP emissions categories.


\begin{figure}[htb]
\centering
\includegraphics[width=\textwidth]{./graphics/bc_prob_gwp.pdf}
\caption{Short-lived climate pollution from open burning of biomass as reported by CARB criteria pollutant emissions inventory.\label{tab:bc_pm}}
\end{figure}

\subsection{Estimating GHG Emissions from Biomass Burning}
\label{sec-2-2}
To estimate GHG emissions from biomass combustion, we estimates pile, wildfire, and
slash burning emissions using published values from CARB(which ones?). 
\begin{itemize}
\item To estimate GHG emissions from \textbf{pile burning}, we use the ratio of
\end{itemize}
PM 2.5 to CO2 and to CH4 from the Piled Fuels
Emissions Calculator. The following ratios are applied to CARB-reported PM emissions to estimate GHG emissions (Table \ref{tab:pfe_calc}).


\begin{center}
\begin{tabular}{rrrrr}
Pile Biomass (t) & Consumed Biomass (t) & PM2.5 (t) & CO2 (t) & CH4 (t)\\
\hline
1.360178 & 1.224161 & 0.008263 & 2.0366 & 0.0034\\
\end{tabular}
\end{center}
\begin{itemize}
\item To estimate GHG emissions from \textbf{wildfire} and \textbf{slash burning}, we In addition, the
\end{itemize}
[[\url{http://www.arb.ca.gov/cc/inventory/archive/tables/net_co2_flux_2007-11-19.pdf}][CARB
greenhouse gas emissions inventory]] estimates emissions from
wildfire and slash burning(<- peter, how is this different to pile burns?) up through 2004 (Table \ref{arb_ghg_2004}) in MMT CO2e.
\begin{table}[htb]
\caption{Annual GHG Emissions estimated from CARB GHG emissions inventory \label{arb_ghg_2004}}
\centering
\begin{tabular}{lr}
Source Category & Average annual emissions 1994-2004 MMTCO$_{\text{2e}}$\\
\hline
Forest and rangeland fires & 2.0194\\
Timber harvest slash & 0.155266666666667\\
\end{tabular}
\end{table}

\subsection{Estimating Total Emissions from Biomass Burning}
\label{sec-2-3}
To arrive at an estimate of total CO2 equivalent (CO2e) emissions in 2015 from burning of forest
management residuals using published CARB
estimates, we combine the CO2 emissions reported for 2004 \emph{\emph{\emph{in the
LULUC <-- where is this from??? Biodegradable Carbon Emissions and Sinks}}} with black carbon
emissions extrapolated from the CARB Criteria Air Pollutant Emissions
inventory estimates. The time discrepancy between the 2004 and 2015 is
acknowledged as an irreconcilable source of uncertainty in this
estimation. Further model-based estimation could be used to derive a
ratio of GHG to PM using the \href{http://www.fs.fed.us/pnw/fera/research/smoke/consume/index.shtml}{USFS CONSUME} model. This analysis does however show that a baseline of
substantial emissions from forest management residuals has been reported
in CARB emissions inventories and should be recognized as a baseline
condition. We find that a rough estimate of CO2e emissions from pile
burning annual approaches 1 Mt CO2e \ref{tab:carb1990_co2e}.

\begin{table}[htb]
\caption{Estimated average annual CO2 equivilant emissions by source and emission type. \label{carb1990_co2e}}
\centering
\begin{tabular}{lr}
sc$_{\text{cat}}$ & avg(mmtco2e)\\
\hline
Forest and rangeland fires & 2.0194\\
Timber harvest slash & 0.155266666666667\\
\end{tabular}
\end{table}

\begin{center}
\begin{tabular}{rrl}
 & Mt CO2e & Source\\
\hline
0 & 0.17 & CO2 pile burning\\
1 & 0.99 & CO2e BC pile burning\\
2 & 1.16 & Total Mt CO2e\\
\end{tabular}
\end{center}

\emph{\emph{BC emissions in terms of CO2e has not been included in any GHG emissions
inventory published by CARB. Remove?}}

\section{Estimating Emissions Impact from Utilization of Harvested Wood}
\label{sec-3}
Wood harvested from California's forests are utilized in a variety of construction,
landscaping, and consumer products. During the manufacture of these products, this wood is fractionated 
through a multi-stage process of harvesting, processing, and utilization to reside in several residual biomass fates (below) for which the time horizon of carbon return to the atmosphere varies widely. 

\begin{description}
\item[{Logging Residuals}] Tops, limbs, and sub-merchantable material produced from harvest activities in the woods
\item[{Processing (Mill) Residuals}] Sawdust, shavings, bark, and off cuts from primary and secondary manufacturing.
\item[{Construction Debris}] Fraction of wood used in construction or  finished products that are not integratrated into its final form.
\item[{Demolition}] Wood used in construction that has reached the end of its useful life.
\end{description}

Each category has multiple potential fates which can greatly influence the net emissions impact attributed to the initial forest management activity. While wood products used in construction or finished products may sequester carbon in a stable environment for a long period, residues sent to landfills or left in the woods as slash emit climate forcing gasses. These wood residues may be directed towards alternative product streams (i.e., chip, power and heat generation) with controlled combustion \emph{\emph{peter - I changed the  meaning here}}. The fate of each of these pools is determined by a highly dynamic political and economic system. To understand how policy decisions will impact the fate and subsequent climate impact of harvested wood products, a detailed process model is necessary. \ref{wood_fates}

\subsection{Disposition of Harvested Wood in California.}
\label{sec-3-1}
To provide a rough estimate of the fate of annually harvested roundwood material, we apply what we know about milling efficiency improvements, logging utilization rates, and construction use efficiency to historical production volumes. 

\subsubsection{Logging Residues}
\label{sec-3-1-1}
According to \citet{Morgan}, logging residues produced from sawlog harvest can be estimated using a factor of 0.0302 (+/-.0123 @95\%CI) times the total cubic sawlog volume delivered to a mill. Unfortunately, we cannot say how logging residue production has changed over time in California.  \citet{Simmons2014} found that logging utilization has decreased in Idaho from 1990 to 2011 by 72\%. For the purpose of this analysis, we will assume that similar changes have occurred in California timber harvesting. We then estimate a logging residue production factor for years before 1990 based on the following equation. We assume 1990 residue ratios for all years prior.


\begin{align*}
V\llap{--}lr_{x} = V\llap{--}rw_{x}\left(\eta_{04}+\left(\eta_{o4}\eta_\Delta\right)\right)\\
\text{Where:}\\
V\llap{--}rw_{x} = \text{Rundwood volume harvested in year }x\\
\eta_{04} = \mathcal{N}(0.0302,0.0123) \text{ ratio of logging residues to roundwood harvested in CA, 2004}\\
\eta_\Delta = 0.72 \text{ (percent change in efficiency over time period)}\\
\end{align*}

For logging residue production factors between 1990 and 2004, we calculate logging residues as a function of the percent change in logging residual ratios estimated for Idaho \citet{Simmons2014} applied to the known logging residual ratio reported by \citet{Morgan}. To reflect the uncertainty in the estimate provided by \citet{Morgan}, we calculate the logging residual using a randomly selected value from a normal probability distribution defined by the estimate and upper and lower bounds of the 95\% confidence interval provided:


\begin{align*}
V\llap{--}lr_{x} = V\llap{--}rw_{x}\left(\eta_{04}+ \left(\eta_{04}\left(\left(Y_1-x\right)\frac{\eta_\Delta}{Y_\Delta}\right)\right)\right)\\
\text{Where:}\\
V\llap{--}rw_{x} = \text{Rundwood volume harvested in year }x\\
\eta_{04} = \mathcal{N}(0.0302,0.0123) \text{ ratio of logging residues to roundwood harvested in CA, 2004}\\
Y_1 = 2004 \text{ (year for which logging residual estimate available for CA)} \\
x = \text{year for which logging residues are calculated}\\
\eta_\Delta = 0.72 \text{ (percent change in logging residue ratio over time period)}\\
Y_\Delta = 21\text{ (number of years over which logging residue ratio decreased)}
\end{align*}

Logging residual volume in years following 2004 are calculated as follows:

\begin{align*}
V\llap{--}lr_{x} = V\llap{--}rw_{x}\left(\eta_{04}- \left(\eta_{04}\left(\left(x-Y_1\right)\frac{\eta_\Delta}{Y_\Delta}\right)\right)\right)\\
\text{Where:}\\
V\llap{--}rw_{x} = \text{Rundwood volume harvested in year }x\\
\eta_{04} = \mathcal{N}(0.0302,0.0123) \text{ ratio of logging residues to roundwood harvested in CA, 2004}\\
Y_1 = 2004 \text{ (year for which logging residual estimate available for CA)} \\
x = \text{year for which logging residues are calculated}\\
\eta_\Delta = 0.72 \text{ (percent change in logging residue ratio over time period)}\\
Y_\Delta = 21\text{ (number of years over which logging residue ratio decreased)}
\end{align*}

\subsubsection{Processing Residues}
\label{sec-3-1-2}
Milling efficiency has increased by roughly 14\% in California in the period between 1970 and 2006 \citet{Keegan2010}. For this analysis we assume a continuous improvement such that for years prior to 1970, milling efficiency in year $x$ is calculated as:


\begin{align*}
V\llap{--}mr_{x} = V\llap{--}rw_{x} \left(\eta_{70}-\left((Y_1-x)\frac{\eta_\Delta}{Y_\Delta}\right\right)\\
\text{Where:}\\
V\llap{--}rw_{x} = \text{Rundwood volume harvested in year }x\\
\eta_{70} = 0.42 \text{ (milling efficiency in 1970)}\\
Y_1 = 1970 \text{ (earliest year mill efficiency available for)} \\
x = \text{year for which milling residues are calculated}\\
\eta_\Delta = 0.06\text{ (increase in milling efficiency from 1970-2011)}\\
Y_\Delta = 41\text{ (number of years overwhihc milling efficiency increased)}
\end{align*}

For years after 1970, milling efficiency for year $x$ is calculated as:

\begin{align*}
V\llap{--}mr_{x} = V\llap{--}rw_{x} \left(\eta_{70}+\left((x-Y_1)\frac{\eta_\Delta}{Y_\Delta}\right\right)\\
\text{Where:}\\
V\llap{--}rw_{x} = \text{Rundwood volume harvested in year }x\\
\eta_{70} = 0.42 \text{ (milling efficiency in 1970)}\\
Y_1 = 1970 \text{ (earliest year mill efficiency available for)} \\
x = \text{year for which milling residues are calculated}\\
\eta_\Delta = 0.06\text{ (increase in milling efficiency from 1970-2011)}\\
Y_\Delta = 41\text{ (number of years overwhihc milling efficiency increased)}
\end{align*}

\subsubsection{Construction Residues}
\label{sec-3-1-3}
To estimate annualized construction waste material, we use ratios of finished wood products to construction debris and demolition debris referenced in \citet{McKeever2004}. This data from \citeauthor{McKeever2004} is sparse and should be considered unreliable for years other than those for which it is reported.  Construction debris was estimated in 2002 as approximately 15\% of the total wood used in construction. Demolition debris from wood produced annually from wood grown on California forestland is outside of the scope of this report.

Table \ref{tab:me_and_lr} presents ten year average estimates of logging and milling residuals, finished lumber, and construction debris based on Board of Equalization (BOE) roundwood harvest volumes. \emph{/ this is kinda random? no mention of BOE data earlier?/}

\begin{longtable}{rrrrrrr}
\caption{Ten-year average logging and mill residual estimates based on BOE harvest volumes in Million Cubic Feet (MCF). RW:Roundwood harvested, LR: Logging residues, MR: Mill Residues, FL: Finished Lumber, CD: Construction Debris}
\\
10-year start & 10-year end & RW & LR & MR & FL & CD\\
\hline
\endhead
\hline\multicolumn{7}{r}{Continued on next page} \\
\endfoot
\endlastfoot
1978 & 1988 & 681.701 & 33.4221 & 299.522 & 382.179 & 57.3269\\
1979 & 1989 & 680.582 & 38.5144 & 300.229 & 380.353 & 57.0529\\
1980 & 1990 & 681.083 & 29.0056 & 301.528 & 379.555 & 56.9333\\
1981 & 1991 & 681.601 & 38.4843 & 302.612 & 378.989 & 56.8483\\
1982 & 1992 & 686.631 & 34.0527 & 305.606 & 381.025 & 57.1538\\
1983 & 1993 & 695.872 & 32.5119 & 310.422 & 385.451 & 57.8176\\
1984 & 1994 & 678.459 & 29.0421 & 303.4 & 375.059 & 56.2589\\
1985 & 1995 & 657.737 & 22.0154 & 294.892 & 362.845 & 54.4267\\
1986 & 1996 & 631.918 & 29.7064 & 284.093 & 347.825 & 52.1738\\
1987 & 1997 & 600.752 & 28.6345 & 270.919 & 329.833 & 49.4749\\
1988 & 1998 & 560.495 & 30.6181 & 253.572 & 306.923 & 46.0384\\
1989 & 1999 & 518.282 & 25.4513 & 235.308 & 282.975 & 42.4462\\
1990 & 2000 & 477.206 & 16.0024 & 217.442 & 259.764 & 38.9645\\
1991 & 2001 & 436.798 & 18.6401 & 199.72 & 237.078 & 35.5618\\
1992 & 2002 & 411.648 & 17.4221 & 188.838 & 222.81 & 33.4214\\
1993 & 2003 & 389.756 & 15.4972 & 179.386 & 210.37 & 31.5555\\
1994 & 2004 & 370.287 & 11.6635 & 171.013 & 199.274 & 29.8912\\
1995 & 2005 & 360.411 & 11.0544 & 166.982 & 193.429 & 29.0143\\
1996 & 2006 & 349.131 & 12.025 & 162.271 & 186.86 & 28.0291\\
1997 & 2007 & 338.319 & 12.9357 & 157.756 & 180.563 & 27.0845\\
1998 & 2008 & 321.14 & 13.1483 & 150.231 & 170.909 & 25.6364\\
1999 & 2009 & 299.649 & 11.5494 & 140.54 & 159.109 & 23.8663\\
2000 & 2010 & 283.222 & 7.59818 & 133.256 & 149.966 & 22.4949\\
2001 & 2011 & 271.892 & 8.49812 & 128.347 & 143.545 & 21.5318\\
2002 & 2012 & 266.945 & 7.48753 & 126.396 & 140.549 & 21.0823\\
2003 & 2013 & 266.193 & 6.92275 & 126.488 & 139.705 & 20.9558\\
2004 & 2014 & 262.901 & 6.76711 & 125.34 & 137.561 & 20.6341\\
\end{longtable}

\subsection{Emissions from un-utilized logging residues}
\label{sec-3-2}

From logging residuals not used in bioenergy, emmisions are produced
from combustion or biological decomposition of the
material over time. To calculate the ratio of burned to decompsed
logging residues, I first calculate the total biomass volume of pile burned forest management residuals, then compare with total residues as reported by the TPO to find the difference. C02e emissions are then independly derived for both conversion streams of logging residuals. \emph{/} forremove  utilize the CARB estimate of annual PM2.5 emissions produced from
forest management with the Consume fire behavior model to extrapolate total 

\begin{enumerate}
\item Estimate biomass from PM2.5:
\label{sec-3-2-0-1}

To estimate total biomass from PM2.5, I assume 90\% consumption of biomass in piles and use the relationship of pile tonnage to PM emissions calculated using the \href{http://depts.washington.edu/nwfire/piles/}{Piled Fuels Biomass and Emissions Calculator} provided by the Washington State Department of Natural Resources. This calculator is based on the \href{http://www.fs.fed.us/pnw/fera/research/smoke/consume/index.shtml}{Consume} fire behavior model published by the US Forest Service. The ratio of PM2.5 to unburned tonnage of biomass used below is \texttt{0.00605508984853}. Ratio of PM2.5 to consumed fuel is \texttt{0.00672787321276}.


\begin{table}[htb]
\caption{Forest biomass burned in piles based on ARB-reported PM2.5 emissions in the 'Forest Management' category using a ratio of \texttt{164.610674089} ton biomass per ton PM2.5.}
\centering
\begin{tabular}{rrr}
YEAR & PM2.5 (t) & Pile-Burned Biomass (t)\\
\hline
2000 & 5474.31 & 901129.28\\
2005 & 5474.31 & 901129.28\\
2010 & 5474.31 & 901129.28\\
2012 & 5477.3 & 901621.96\\
2015 & 5480.51 & 902150.69\\
\end{tabular}
\end{table}

Total emissions resulting from \textbf{pile burned} forest management residuals
can then be derived for the two greenhouse gasses produced from pile
burning (CO2, CH4) and from BC:


\item Emissions from decomposition of un-utilized forest management residuals:
\label{sec-3-2-0-2}

Un-utilized residual biomass not consumed in pile burns decomposes over
time resulting in methane and carbon dioxide \emph{\emph{inconsistent}} emissions. To provide a
full picture of the emissions from residual material produced from
commercial timber harvesting in California, we must account for decomposition 
of unutilized logging residuals left on-site that are not burned. To establish
 the fraction of logging residue that is left to
decompose, residues burned and used in bioenergy are subtracted from the
total reported by the TPO:

\begin{align*}
LR_d &= LR - LR_{piles} - LR_{bio} \\
\text{where:}\\
LR_d &= \text{Logging residuals subject to anerobic decomposition} \\
LR &= \text{Total logging residue reported by TPO}\\
LR_{piles} &= \text{Logging residues combusted in anthropogenic pile burns}\\
LR_{bio} &= \text{Logging residues used to produce bioenergy}
\end{align*}
To calculate the GHG emissions from decomposition of piles, we use the
following equation.

\begin{align*}
CO_2e_{decomp} &= \left(LR_d \times C_{LR} \times CO2_{ratio} \right) + \left(LR_d \times C_{LR} \times CH_4_{ratio}\times GWP_{CH_4}\right)\\
\text{where:}\\
CO_2e_{decomp} &= \text{Carbon dioxide equivalent emissions from decomposition of logging slash}\\
C_{LR} &= \text{Carbon fraction of biomass: 0.5}\\
CO2_{ratio} &= \text{Fraction of carbon released as } CO_2\text{: 0.61}\\
CH_4_{ratio} &= \text{Fraction of carbon released as } CH_4\text{: 0.09}\\
GWP_{CH_4} &= \text{Global warming potential of methane: 56}
\end{align*}
\end{enumerate}

\subsection{Emissions from non-commercial management residuals}
\label{sec-3-3}


The Timber Products Output (TPO) in California does not report wood volume produced from
non-commercial management activities. This includes management
activities such as pre-commercial thinning, sanitation thinning, and
fuels reduction thinning. To estimate the volume of material produced
from these activities we use the following sources:

\begin{enumerate}
\item \textbf{Public lands:} The USFS Forest Service Activity Tracking System
(FACTS) reports management activities conducted on National Forest
System Lands. To ensure estimates of biomass volume using FACTS are
not duplicative of reported volume in the TPO a series of filters are
applied to the FACTS attributes to identify only non-commercial
management activities.
\end{enumerate}

\begin{enumerate}
\item Forest Service Activity Tracking System (FACTS)
\label{sec-3-3-0-1}

Data from TPO does not account for forest management activities that do
not result in commercial products (timber sales, biomass sales). The
USFS
\href{http://data.fs.usda.gov/geodata/edw/datasets.php?dsetParent=Activities}{reports}
Hazardous Fuels Treatment (HFT) activities as well as Timber Sales (TS)
derived from the FACTS database. I use these two data sets to estimate
the number of acres treated that did not produce commercial material
(sawlogs or biomass) and where burning was not used. The first step is
to eliminate all treatments in the HFT data set that included timber
sales. I accomplish this by eliminating all rows in the HFT data set
that have identical \texttt{FACTS\_ID} fields in the TS dataset. I further
filter the HFT dataset by removing any planned but not executed
treatments (\texttt{nbr\_units1 >0} below -- \texttt{nbr\_units1} references
\texttt{NBR\_UNITS\_ACCOMPLISHED} in the USFS dataset, see metadata for HFT
\href{http://data.fs.usda.gov/geodata/edw/edw_resources/meta/S_USA.Activity_HazFuelTrt_PL.xml}{here}),
and use text matching in the 'ACTIVITY' and 'METHOD' fields to remove
any rows that contain reference to 'burning' or 'fire'. Finally, we
remove all rows that that reference 'Biomass' in the method category as
it is assumed that this means material was removed for bioenergy.I use a
range of 10-35 BDT/acre to convert acres reported in FACTS to volume.
The following table presents descriptive statistics for estimates of
residual unutilized wood biomass on an annual basis in million cubic
feet.

\begin{center}
\begin{tabular}{lrrrrr}
 & nf$\backslash$$_{\text{n}}$ & nf$\backslash$$_{\text{lr}}$ & opriv$\backslash$$_{\text{lr}}$ & fi$\backslash$$_{\text{lr}}$ & opub$\backslash$$_{\text{lr}}$\\
\hline
count & 11 & 4 & 4 & 4 & 4\\
mean & 12.0194 & 17.7 & 28.95 & 66.425 & 2.4\\
std & 4.68948 & 5.07346 & 16.1593 & 6.07639 & 1.79444\\
min & 2.37421 & 11.2 & 11.2 & 59.6 & 0.3\\
25\% & 8.92407 & 15.025 & 19.525 & 62.225 & 1.275\\
50\% & 13.3557 & 18.5 & 27.75 & 66.85 & 2.5\\
75\% & 14.5349 & 21.175 & 37.175 & 71.05 & 3.625\\
max & 17.8532 & 22.6 & 49.1 & 72.4 & 4.3\\
\end{tabular}
\end{center}


\begin{enumerate}
\item \textbf{Private industrial timber lands:} CalFIRE's
\href{http://www.calfire.ca.gov/resource_mgt/resource_mgt_forestpractice_gis}{Forest
Practice Geographical Information System}. \textbf{TODO}
\end{enumerate}
\end{enumerate}

\subsection{Avoided Emissions from Wood Product Displacement Factors}
\label{sec-3-4}

For each product application, wood may be substituted by a range of other materials. For example, in
residential construction, precast concrete and structural steel framing
are competitive alternatives to wood. This choice of materials has a profound impact on GHG emissions in the
construction sector and is expressed as a displacement
factor (DF). A displacement factor quantifies the amount of emissions
reduction achieved per unit of wood used. The displacement factors published in
\citep{Sathre2010} and used in this analysis are based on the
following emission reduction sources:

\begin{enumerate}
\item \textbf{Reduced emissions from manufacturing:} Wood products require less total
energy than to manufacture than products made from alternative materials.
\item \textbf{Avoided process emissions:} Production of wood alternatives such as cement are associated with 
substantial CO2 emissions.
\item \textbf{Carbon storage in products:} Carbon in harvested wood is drawn from
the atmosphere through photosynthesis and will remain fixed through
the useful life of the wood product.
\item \textbf{Carbon storage in forests:} Forests producing wood continue to grow.
It is assumed that forests producing wood in California are managed
to sustain forest growth (not converted to non-forest land uses).
\item \textbf{Avoided fossil fuel emissions due to bioenergy substitution:}
Logging and milling residuals used to produce energy avoid emissions
from fossil energy sources in the energy sector.
\item \textbf{Carbon dynamics in landfills:} A fraction of carbon from wood
deposited in landfills remains in semi-permanent storage.
The remainder is converted to methane through biological
decomposition in the landfill. Capture and use of the methane as an
energy source, in turn reduces emissions from fossil energy sources.
\end{enumerate}

A meta analysis conducted by \citep{Sathre2010} compared empirical analysis from 21 international studies and found an
average emissions reduction of 2.1 tons of carbon (3.9 t CO2e) per ton
of dry wood used. While studies ranged substantially around the average, the
authors found that the majority of published displacement factors ranged
between 1 and 3 tC/t dry wood. 

\subsection{Displacement Factors Applied to Timber Products Output}
\label{sec-3-5}

To evaluate the climate impact of harvested wood in California, I used
harvested roundwood estimates from the Timber Products Output (TPO)
database\footnote{Timber Products Output Reporting Tool \href{http://srsfia2.fs.fed.us/php/tpo_2009/tpo_rpa_int1.php}{\url{http://srsfia2.fs.fed.us/php/tpo_2009/tpo_rpa_int1.php}}}. I used two estimates of the DF applied
to the harvested wood reported in the TPO based on whether logging
residuals were used in bioenergy or left in the woods (to decompse or
burn).

Figure \ref{fig:flow_chart} reflects the flow of wood
from Californias forest to its fate in-use and is the frame of
reference for the following analysis.

\begin{figure}[htb]
\centering
\includegraphics[width=0.75\textwidth]{./graphics/flow_chart.pdf}
\caption{Wood flows from timber harvest in California \label{fig:flow_chart}}
\end{figure}


I applied displacement factors reported by \cite{Sathre2010} to the
reported harvest volumes from the TPO database. 


The following references are used to
arrive at an average displacement factor of \textbf{2.625} tCO2e/t finished
wood product for harvested roundwood without
logging residue utilization.

\begin{table}[htb]
\caption{Wood displacement factor without residue utilization \label{tab:df_no_use}}
\centering
\begin{tabular}{lr}
reference & displacement factor\\
\hline
\citet{Eriksson2007} & 1.7\\
\citet{Eriksson2007} & 2.2\\
\citet{Salazar2009} & 4.9\\
\citet{Werner2005} & 1.7\\
\end{tabular}
\end{table}

For harvested roundwood with logging residue utilization the following
studies are used. I used an average of the DF reported here of \textbf{3.243} tCO2e/t finished
wood product.


\begin{table}[htb]
\caption{Wood discplacement factor with residue utilization \label{tab:df_inc_use}}
\centering
\begin{tabular}{lr}
reference & displacement factor\\
\hline
\citet{Eriksson2007} & 1.9\\
\citet{Eriksson2007} & 2.5\\
\citet{Gustavsson2006a} & 4\\
\citet{Gustavsson2006a} & 5.6\\
\citet{Gustavsson2006a} & 2.2\\
\citet{Gustavsson2006a} & 3.3\\
\citet{Pingoud2001} & 3.2\\
\end{tabular}
\end{table}



The TPO reports values in terms of roundwood harvested for products, but the
displacement factors presented in Sathre and O'Connor are in terms of
tons of carbon in wood products. Therefore we must assume a milling
efficiency to convert TPO volume estimates to finished wood product volume. I assumed
a milling efficiency of 0.5.


Further, TPO is reported in cubic feet and the DF implies a mass
unit. To convert cubic meters to a mass unit, we used the average wood
density of harvested volume in California weighted by species as reported 
in \citet{Mciver2012}. The resulting weighted average wood density used here is \textbf{27.94
lbs/cuft}.


Using the McIver and Morgan, we determine the percent of harvested wood used in bioenergy
feedstocks. From personal communications with
\href{http://www.bber.umt.edu/staff/mciver.asp}{Chelsea McIver}, all bioenergy feedstock reported is sourced in-woods (ie, not mill
residues).

\begin{table}[htb]
\caption{\% volume of wood diverted to Bioenergy use by year \label{tab:bio_vol}}
\centering
\begin{tabular}{rrr}
 & year & bioenergy \% of harvest\\
\hline
0 & 2000 & 0.024\\
1 & 2006 & 0.036\\
2 & 2012 & 0.082\\
\end{tabular}
\end{table}



\begin{center}
\begin{tabular}{rlrrr}
 & Ownership & Roundwood Products & Logging Residues & Year\\
\hline
0 & National Forest & 72.4 & 20.7 & 2012\\
1 & Other Public & 16.2 & 3.4 & 2012\\
2 & Forest Industry & 328.9 & 72.4 & 2012\\
3 & Other Private & 53 & 11.2 & 2012\\
4 & National Forest & 52.8 & 16.3 & 2006\\
5 & Other Public & 1.1 & 0.3 & 2006\\
6 & Forest Industry & 274.3 & 59.6 & 2006\\
7 & Other Private & 139.2 & 33.2 & 2006\\
8 & National Forest & 90.8 & 22.6 & 2000\\
9 & Other Public & 5.2 & 1.6 & 2000\\
10 & Forest Industry & 372.5 & 70.6 & 2000\\
11 & Other Private & 159.4 & 49.1 & 2000\\
12 & National Forest & 132.1 & 11.2 & 1994\\
13 & Other Public & 24.7 & 4.3 & 1994\\
14 & Forest Industry & 396.1 & 63.1 & 1994\\
15 & Other Private & 174.7 & 22.3 & 1994\\
\end{tabular}
\end{center}


In addition to the TPO, the California Board of Equalization (BOE) also
reports historic timber harvest volumes.  Comparing between years where both
sources report data, the BOE database reports on average, 8\% less volume than the TPO (Table \ref{tab:tpo_boe}) database. This is reasonable considering that:
\begin{enumerate}
\item BOE data may be under-reported, as there may be a financial incentive to reduce tax burden
\item BOE does not include volume harvested from native American tribal lands in the state
\end{enumerate}

\begin{longtable}{rrrr}
\caption{Total annual harvest reported by \citet{Mciver2012} and California Board of Equalization.\label{tab:tpo_boe}}
\\
year & McIver, et. al. (2012) MMBF & BOE MMBF & BOE/M\&M\\
\hline
\endhead
\hline\multicolumn{4}{r}{Continued on next page} \\
\endfoot
\endlastfoot
1978 & 4606.0 & 4491 & 0.98\\
1979 & 4044.0 & 3991 & 0.99\\
1980 & 3478.0 & 3164 & 0.91\\
1981 & 2832.0 & 2672 & 0.94\\
1982 & 2488.0 & 2318 & 0.93\\
1983 & 3638.0 & 3358 & 0.92\\
1984 & 3701.0 & 3546 & 0.96\\
1985 & 4093.0 & 3818 & 0.93\\
1986 & 4416.0 & 4265 & 0.97\\
1987 & 4667.0 & 4500 & 0.96\\
1988 & 4847.0 & 4670 & 0.96\\
1989 & 4699.0 & 4424 & 0.94\\
1990 & 4264.0 & 4021 & 0.94\\
1991 & 3439.0 & 3195 & 0.93\\
1992 & 3192.0 & 2973 & 0.93\\
1993 & 3041.0 & 2871 & 0.94\\
1994 & 2814.0 & 2316 & 0.82\\
1995 & 2520.0 & 2306 & 0.92\\
1996 & 2515.0 & 2273 & 0.9\\
1997 & 2640.0 & 2400 & 0.91\\
1998 & 2420.0 & 2091 & 0.86\\
1999 & 2429.0 & 2144 & 0.88\\
2000 & 2244.0 & 1966 & 0.88\\
2001 & 1801.0 & 1603 & 0.89\\
2002 & 1691.73 & 1690 & 1.0\\
2003 & 1667.95 & 1663 & 1.0\\
2004 & 1704.0305 & 1706 & 1.0\\
2005 & 1738.5 & 1725 & 0.99\\
2006 & 1960.35 & 1631 & 0.83\\
2007 & 1759.6 & 1626 & 0.92\\
2008 & 1476.0745 & 1372 & 0.93\\
2009 & 911.19 & 805 & 0.88\\
2010 & 1302.38 & 1161 & 0.89\\
2011 & 1432.5 & 1288 & 0.9\\
2012 & 1421.3 & 1307 & 0.92\\
\end{longtable}

// move to appendix?//The TPO reports harvest from tribal lands, which produces an average 0.74\% of the total
annual harvest in the state for the 37 years of parallel data. For
this analysis we used TPO data to include harvest volume from tribal lands. 


\begin{longtable}{rrrrr}
\caption{Annual harvest by ownership from \citet{Mciver2012} (MCF)\label{tab:MandM}}
\\
year & State & Federal & Private & Tribal\\
\hline
\endhead
\hline\multicolumn{5}{r}{Continued on next page} \\
\endfoot
\endlastfoot
1947 & 0.0 & 0.0 & 569.85 & 0.0\\
1948 & 0.0 & 0.0 & 735.29 & 0.0\\
1949 & 0.0 & 0.0 & 698.53 & 0.0\\
1950 & 0.0 & 0.0 & 808.82 & 0.0\\
1951 & 0.0 & 0.0 & 900.74 & 0.0\\
1952 & 2.57 & 113.79 & 808.82 & 4.78\\
1953 & 3.31 & 117.65 & 977.94 & 2.76\\
1954 & 2.94 & 141.54 & 880.51 & 4.6\\
1955 & 2.57 & 191.73 & 906.25 & 6.07\\
1956 & 4.41 & 206.99 & 862.13 & 5.33\\
1957 & 4.96 & 170.59 & 801.47 & 6.62\\
1958 & 5.51 & 208.27 & 821.69 & 6.99\\
1959 & 4.96 & 279.6 & 788.6 & 9.19\\
1960 & 5.15 & 250.37 & 680.15 & 8.82\\
1961 & 5.33 & 259.74 & 707.72 & 10.11\\
1962 & 6.25 & 259.01 & 744.49 & 8.64\\
1963 & 4.04 & 311.76 & 678.31 & 9.93\\
1964 & 4.6 & 348.16 & 643.38 & 9.01\\
1965 & 5.7 & 363.05 & 591.91 & 9.74\\
1966 & 5.88 & 360.85 & 545.96 & 8.27\\
1967 & 6.43 & 355.51 & 562.5 & 7.54\\
1968 & 8.82 & 440.44 & 542.28 & 14.52\\
1969 & 7.35 & 372.61 & 529.41 & 9.93\\
1970 & 6.25 & 345.4 & 481.62 & 5.15\\
1971 & 7.17 & 383.09 & 476.1 & 12.87\\
1972 & 6.8 & 411.58 & 591.91 & 12.13\\
1973 & 6.07 & 371.69 & 516.54 & 9.38\\
1974 & 7.35 & 322.79 & 525.74 & 9.38\\
1975 & 6.43 & 287.87 & 498.16 & 3.31\\
1976 & 7.35 & 348.53 & 507.35 & 6.99\\
1977 & 5.15 & 323.35 & 544.12 & 6.99\\
1978 & 5.15 & 332.35 & 509.19 & 8.64\\
1979 & 4.78 & 321.32 & 417.28 & 8.82\\
1980 & 3.68 & 279.04 & 356.62 & 7.72\\
1981 & 2.76 & 201.65 & 316.18 & 4.04\\
1982 & 7.72 & 173.9 & 275.74 & 1.47\\
1983 & 7.9 & 313.42 & 347.43 & 2.57\\
1984 & 6.25 & 288.05 & 386.03 & 3.86\\
1985 & 6.62 & 339.52 & 406.25 & 0.92\\
1986 & 5.33 & 365.26 & 441.18 & 4.96\\
1987 & 7.72 & 364.89 & 485.29 & 7.54\\
1988 & 5.7 & 403.68 & 481.62 & 2.57\\
1989 & 6.8 & 373.53 & 483.46 & 2.02\\
1990 & 4.41 & 283.09 & 496.32 & 2.57\\
1991 & 6.99 & 248.35 & 376.84 & 4.41\\
1992 & 4.23 & 190.99 & 391.54 & 5.88\\
1993 & 6.25 & 137.32 & 415.44 & 2.39\\
1994 & 3.12 & 152.02 & 362.13 & 2.76\\
1995 & 7.35 & 101.1 & 354.78 & 2.94\\
1996 & 10.11 & 86.4 & 365.81 & 2.39\\
1997 & 8.64 & 101.65 & 375.0 & 2.76\\
1998 & 4.78 & 83.46 & 356.62 & 2.94\\
1999 & 0.0 & 97.24 & 349.26 & 0.0\\
2000 & 3.49 & 63.42 & 345.59 & 1.84\\
2001 & 2.94 & 56.07 & 272.06 & 1.84\\
2002 & 0.18 & 31.38 & 279.41 & 2.5\\
2003 & 0.18 & 28.85 & 277.57 & 3.29\\
2004 & 0.18 & 20.78 & 292.28 & 3.05\\
2005 & 0.18 & 43.66 & 275.74 & 1.95\\
2006 & 0.74 & 41.61 & 318.01 & 2.37\\
2007 & 0.18 & 58.57 & 264.71 & 3.55\\
2008 & 0.18 & 37.7 & 233.46 & 2.48\\
2009 & 0.18 & 30.37 & 136.95 & 0.72\\
2010 & 0.18 & 49.89 & 189.34 & 1.79\\
2011 & 0.18 & 55.42 & 207.72 & 2.1\\
2012 & 5.13 & 37.39 & 218.75 & 1.49\\
\end{longtable}

To use the TPO data to estimate emissions reductions using the DF, we apply a
conversion factor of \textbf{5.44} MCF/MMBF. This is an approximation as the
actual sawlog conversion factor varies with average harvested log size, which has changed over time.  


Using the ratio of logging residuals consumed by bioenergy (mciver), to the total logging residuals reported in the TSP, we can calculated the harvest volume the ratio of harvest volume to logging residuals used in bioenergy,
we calculateted 
based on the ratio of reported consumption of logging residuals in
bioenergy by \citeauthor{Mciver2012} to the total logging residuals reported
in the TPO. \citeauthor{Mciver2012} report bioenergy consumption from 2000
forward. For years previous, we use the average bioenergy consumption
from 2000 -- 2012. These results assume bioenergy consumption
throughout the reporting years. Bioenergy use of residuals did not
begin until the late 1970. Further analysis is necessary to modify
these results to reflect the development of the bioenergy industry.

To calculate the total emissions reduction resulting from California's
timber harvest, we apply the appropriate displacement factor (with or
without logging residual utilization) to the commensurate fraction of
harvested roundwood. The results are shown in the following chart.

\begin{figure}[htb]
\centering
\includegraphics[width=\textwidth]{./graphics/ann_hh_em_reduc.pdf}
\caption{Historical emissions reductions resulting from harvested roundwood using displacement factors from \citep{Sathre2010} applied to TPO data.\label{em_reduc_hist}}
\end{figure}

Contribution of the varios ownership categories to the aggregate is
shown in Figure \ref{em_reduc_own}.

\begin{figure}[htb]
\centering
\includegraphics[width=.9\linewidth]{./graphics/harv_em_reductions.png}
\caption{\label{em_reduc_own}Historical emissions reductions by ownership for selected years resulting from harvested roundwood using displacement factors from \citep{Sathre2010} applied to TPO data. \label{em_reduc_own}}
\end{figure}

\section{Further Questions}
\label{sec-4}

This analysis is a first step towards a broader analysis of the
climate impacts of harvested wood in California. The following are key
questions which follow from this analysis.

\section{References}
\label{sec-5}
\bibliographystyle{IEEEtranSN}
\bibliography{fcat}
% Emacs 24.4.1 (Org mode 8.2.10)
\end{document}
