% Created 2017-01-05 Thu 23:49
\documentclass[a4paper]{article}
\usepackage[utf8]{inputenc}
\usepackage[T1]{fontenc}
\usepackage{fixltx2e}
\usepackage{graphicx}
\usepackage{grffile}
\usepackage{longtable}
\usepackage{wrapfig}
\usepackage{rotating}
\usepackage[normalem]{ulem}
\usepackage{amsmath}
\usepackage{textcomp}
\usepackage{amssymb}
\usepackage{capt-of}
\usepackage{hyperref}
\usepackage{amssymb,amsmath}
\usepackage{natbib}
\usepackage[margin=2cm]{geometry}
\usepackage{fancyhdr} %For headers and footers
\pagestyle{fancy} %For headers and footers
\usepackage{acronym}
\usepackage{lastpage} %For getting page x of y
\usepackage{float} %Allows the figures to be positioned and formatted nicely
\floatstyle{boxed} %using this
\usepackage{draftwatermark}
\restylefloat{figure} %and this command
\usepackage{url} %Formatting of yrls
\rhead{\includegraphics[width=3cm]{berkeley}}
\chead{}
\lfoot{Draft}
\cfoot{}
\setlength{\parskip}{1em}
\rfoot{\thepage\ of \pageref{LastPage}}
\acrodef{GHG}{Greenhouse Gas}
\acrodef{SLCP}{Short-Lived Climate Pollutants}
\acrodef{CAP}{Criteria Air Pollutants}
\acrodef{PM2.5}{Particulate Matter 2.5 $\mu$m}
\acrodef{NOX}{Oxides of Nitrogen}
\acrodef{CO2e}{Carbon Dioxide Equivalents}
\acrodef{CARB}{California Air Resources Board}
\acrodef{DF}{Displacement Factor}
\acrodef{FCP}{Forest Climate Plan}
\acrodef{BOF}{California Board of Forestry}
\acrodef{BC}{Black Carbon}
\acrodef{TC}{Total Carbon}
\acrodef{BOE}{California Board of Equalization}
\acrodef{TPO}{Timber Products Output}
\acrodef{OC}{Organic Carbon}
\author{Peter Tittmann, Ph.D.}
\date{\today}
\title{Emissions from Decomposition of un-utilized forest management residuals}
\hypersetup{
 pdfauthor={Peter Tittmann, Ph.D.},
 pdftitle={Emissions from Decomposition of un-utilized forest management residuals},
 pdfkeywords={},
 pdfsubject={},
 pdfcreator={Emacs 24.5.1 (Org mode 8.3.4)}, 
 pdflang={English}}
\begin{document}

\maketitle
\tableofcontents

Un-utilized residual biomass not consumed in pile burns decomposes over
time resulting in CH4  and CO2 emissions. To provide a
full picture of the emissions from residual material produced from
commercial timber harvesting in California, we must account for decomposition 
of unutilized logging residuals left on-site that are not burned. 

\begin{align*}
LR_d &= LR - LR_{piles} - LR_{bio} \\
\text{where:}\\
LR_d &= \text{Logging residuals subject to anerobic decomposition} \\
LR &= \text{Total logging residue reported by TPO}\\
LR_{piles} &= \text{Logging residues combusted in anthropogenic pile burns}\\
LR_{bio} &= \text{Logging residues used to produce bioenergy}
\end{align*}
To calculate the \ac{GHG} emissions from decomposition of piles, we use the
following equation.

\begin{align*}
CO_2e_{decomp} &= \left(LR_d \times C_{LR} \times CO2_{ratio} \right) + \left(LR_d \times C_{LR} \times CH_4_{ratio}\times GWP_{CH_4}\right)\\
\text{where:}\\
CO_2e_{decomp} &= \text{Carbon dioxide equivalent emissions from decomposition of logging slash}\\
C_{LR} &= \text{Carbon fraction of biomass: 0.5}\\
CO2_{ratio} &= \text{Fraction of carbon released as } CO_2\text{: 0.61}\\
CH_4_{ratio} &= \text{Fraction of carbon released as } CH_4\text{: 0.09}\\
GWP_{CH_4} &= \text{Global warming potential of methane: 56}
\end{align*}

To establish the fraction of logging residue that is left to decompose, residues burned and used in bioenergy are subtracted from total logging residuals produced from roundwood harvest. \cite{Mciver2012} report bioenergy consumption in the state (Table \ref{tab:bio_vol}).

\begin{table}[htb]
\centering
\begin{tabular}{rr}
year & Percent of roundwood harvest used in bioenergy\\
\hline
2000 & 2.4\\
2006 & 3.6\\
2012 & 8.2\\
\end{tabular}
\caption{\% volume of wood diverted to Bioenergy use by year \label{tab:bio_vol}}

\end{table}

The availability of data for bioenergy consumption of logging residuals does not allow us to precisely estimate the consumption for years other than reported by \cite{Mciver2012}. In this analysis, for years that bioenergy consumption is reported, I use that value. As the states biomass energy infrastructure began to consume substantial amounts of residual in the early 1980's \citep{Morris2000}, we assume that the average consumption from the 3 years reported is representative annual consumption. For years before 1980, we assume no bioenergy consumption.  This approach is less than ideal as there has been a great deal of variability in the appetite for logging residuals from biomass power plants. Un-utilized logging residues are estimated from logging residuals not used in bioenergy (Table \ref{tab:unused_lr}). These results are based on a normal probability distribution for logging residual generation from roundwood harvest. This is one of several factors contributing to instances where bioenergy consumption is greater than logging residues  produced. Other factors include:

\begin{itemize}
\item Lack of temporal resolution in bioenergy consumption
\item Consumption by biomass power plants of in-woods residuals produced from forest management that did not result in commercial roundwood harvest
\end{itemize}


\begin{table}[htb]
\centering
\begin{tabular}{rrrr}
Year & Logging Residues & Bioenergy & Unutilized Logging Residuals\\
\hline
1978 & 1.56564 & 0 & 1.56564\\
1979 & 1.2689 & 0 & 1.2689\\
1980 & 0.756188 & 0.348909 & 0.40728\\
1981 & 0.556558 & 0.294654 & 0.261904\\
1982 & 0.320001 & 0.255616 & 0.064385\\
1983 & 1.19463 & 0.370302 & 0.824331\\
1984 & 1.28008 & 0.391033 & 0.889043\\
1985 & 0.834831 & 0.421028 & 0.413803\\
1986 & 1.34222 & 0.470321 & 0.871901\\
1987 & 1.80944 & 0.496235 & 1.3132\\
1988 & 0.275234 & 0.514982 & -0.239748\\
1989 & 1.27182 & 0.487854 & 0.78397\\
1990 & 1.19944 & 0.443414 & 0.756022\\
1991 & 0.641766 & 0.352327 & 0.289439\\
1992 & 0.564071 & 0.327846 & 0.236225\\
1993 & 0.532689 & 0.316598 & 0.216091\\
1994 & 0.571306 & 0.255396 & 0.31591\\
1995 & 0.249442 & 0.254293 & -0.004851\\
1996 & 0.300928 & 0.250654 & 0.0502744\\
1997 & 0.3554 & 0.264659 & 0.0907408\\
1998 & 0.186836 & 0.230584 & -0.0437481\\
1999 & 0.20601 & 0.236429 & -0.0304188\\
2000 & 0.447945 & 0.109927 & 0.338018\\
2001 & 0.234992 & 0.17677 & 0.0582215\\
2002 & 0.357945 & 0.186364 & 0.171581\\
2003 & 0.29514 & 0.183387 & 0.111754\\
2004 & 0.155414 & 0.188128 & -0.0327142\\
2005 & 0.250076 & 0.190224 & 0.0598525\\
2006 & 0.225896 & 0.136793 & 0.0891028\\
2007 & 0.167812 & 0.179306 & -0.0114946\\
2008 & 0.161299 & 0.151297 & 0.0100027\\
2009 & 0.078976 & 0.088771 & -0.009795\\
2010 & 0.150674 & 0.128029 & 0.022645\\
2011 & 0.133867 & 0.142034 & -0.00816644\\
2012 & 0.190052 & 0.249688 & -0.0596362\\
2013 & 0.172895 & 0.181402 & -0.00850613\\
2014 & 0.158662 & 0.161662 & -0.00300087\\
\end{tabular}
\caption{Probabilistic disposition of logging residuals from roundwood harvest in CA. Volume in million bone-dry tons.}

\end{table}

To estimate the emissions from decomposition of logging residuals that are not burned, an estimate of consumption of biomass in pile burns would be necessary. In theory, the \ac{CARB} \ac(CAP\} inventory could provide an estimate using the ratio of \ac{PM} to biomass consumed. However the \ac{CARB}-derived pile burn estimate far exceeds the volume of logging residuals from the \ac{BOE} historical harvest data (Table \ref{tab:pile_bio_comparison})

\begin{table}[htb]
\centering
\begin{tabular}{rr}
\ac{CARB} estimate (BDT) & \ac{BOE} estimate (BDT)\\
\hline
901423.23 & 553110.26\\
\end{tabular}
\caption{Comparison of annual pile-burned biomass from forestry by \ac{CARB} with \ac{BOE}-derived estimate of loggin residuals produced from timber harvest.}

\end{table}

This is likely due to in part to the fact that the \ac{CARB} estimate includes non-commercial forest management activity.
\end{document}