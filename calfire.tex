% Created 2017-01-04 Wed 11:46
\documentclass[a4paper]{article}
\usepackage[utf8]{inputenc}
\usepackage[T1]{fontenc}
\usepackage{fixltx2e}
\usepackage{graphicx}
\usepackage{longtable}
\usepackage{float}
\usepackage{wrapfig}
\usepackage{rotating}
\usepackage[normalem]{ulem}
\usepackage{amsmath}
\usepackage{textcomp}
\usepackage{marvosym}
\usepackage{wasysym}
\usepackage{amssymb}
\usepackage{hyperref}
\tolerance=1000
\usepackage{amssymb,amsmath}
\usepackage{natbib}
\usepackage[margin=2cm]{geometry}
\usepackage{fancyhdr} %For headers and footers
\pagestyle{fancy} %For headers and footers
\usepackage{acronym}
\usepackage{lastpage} %For getting page x of y
\usepackage{float} %Allows the figures to be positioned and formatted nicely
\floatstyle{boxed} %using this
\usepackage{draftwatermark}
\restylefloat{figure} %and this command
\usepackage{url} %Formatting of yrls
\rhead{\includegraphics[width=3cm]{berkeley}}
\chead{}
\lfoot{Draft}
\cfoot{}
\setlength{\parskip}{1em}
\rfoot{\thepage\ of \pageref{LastPage}}
\acrodef{GHG}{Greenhouse Gas}
\acrodef{SLCP}{Short-Lived Climate Pollutants}
\acrodef{CAP}{Criteria Air Pollutants}
\acrodef{PM2.5}{Particulate Matter 2.5 $\mu$m}
\acrodef{NOX}{Oxides of Nitrogen}
\acrodef{CO2e}{Carbon Dioxide Equivalents}
\acrodef{CARB}{California Air Resources Board}
\acrodef{DF}{Displacement Factor}
\acrodef{FCP}{Forest Climate Plan}
\acrodef{BOF}{California Board of Forestry}
\acrodef{BC}{Black Carbon}
\acrodef{TC}{Total Carbon}
\acrodef{BOE}{California Board of Equalization}
\acrodef{TPO}{Timber Products Output}
\acrodef{OC}{Organic Carbon}
\author{Peter Tittmann, Ph.D.}
\date{\today}
\title{Estimating Black Carbon Emissions from Biomass Burning}
\hypersetup{
  pdfkeywords={},
  pdfsubject={},
  pdfcreator={Emacs 24.4.1 (Org mode 8.2.10)}}
\begin{document}

\maketitle
\tableofcontents

\acf{BC} is not directly reported by statewide emissions summaries.\ac{BC} is a fraction of the \ac{TC} component of \ac{PM2.5}. \ac{PM2.5} emissions are published annually by \ac{CARB} (\href{http://www.arb.ca.gov/ei/emissiondata.htm}{Criteria air pollutant (CAP) emissions estimates}). 
By using the 2015 CAP emissions estimates shown in Table \ref{tab:arb_pm_ann} with estimated ratios of 
smoldering to flaming combustion for hand/machine piled burns, prescribed 
natural fire and wildfire from \citet{Ward1989}, Black Carbon emissions
can be calculated from PM
2.5 with Eq. \eqref{eq-bc}


\begin{center}
\begin{tabular}{llr}
Source (\ac{CARB} nomenclature) & Description & PM 2.5 (t y$^{\text{-1}}$)\\
\hline
ALL VEGETATION & Wildfire & 137630.15\\
FOREST MANAGEMENT & Pile burning & 5480.51\\
WILDLAND FIRE USE (WFU) & Prescribed natural fire & 6802.43\\
\end{tabular}
\end{center}

Using the 2015 \ac{CAP} emissions estimates shown in Table \ref{tab:arb_pm_ann} with estimated ratios of smoldering to flaming combustion for hand/machine piled burns, prescribed natural fire and wildfire from \citet{Ward1989}, \ac{BC} emissions can be estimated from PM 2.5 using equation \eqref{eq-bc}


\begin{align}
BC &= \left( PM_{2.5} \times F \times TC_f \times BC_f\right) + \left( PM_{2.5} \times S \times TC_s \times BC_s\right) \label{eq-bc} \\
\text{where:} \nonumber \\
BC &= \text{Black Carbon (mass units)} \nonumber \\
PM_{2.5} &= PM_{2.5} \text{ (mass units)} \nonumber \\
F &= \text{Percent of combustion in flaming phase} \nonumber \\
TC_f &= \text{Total Carbon fraction of } PM_{2.5} \text{ for flaming phase} \nonumber \\
BC_f &= \text{Black Carbon fraction of Total Carbon for flaming phase} \nonumber \\
S &= \text{Percent of combustion in smoldering phase} \nonumber \\
TC_s &= \text{Total Carbon fraction of } PM_{2.5} \text{ for smoldering phase} \nonumber \\
BC_s &= \text{Black Carbon fraction of Total Carbon for smoldering phase} \nonumber
\end{align}




The ratio of smoldering to flaming combustion behavior for each biomass burning scenario means that each has a different \ac{BC} $\Delta$ \ac{PM2.5}
ratio. To arrive at a rough estimate of \ac{BC} emissions based on PM2.5, ratios from  \citet{Ward1989} and \citet{Jenk1996} ratios in Table \ref{tab:bc_pm} are used herein.
\begin{table}[htb]
\caption{Factors used for calculating \ac{BC} emissions from the three primary combustion sources. \ac{BC} is a fraction of \ac{TC} which is a fraction of total \ac{PM2.5}. Coefficients of variation (C$_{\text{v}}$) are reported here as well. \label{tab:bc_pm}}
\centering
\begin{tabular}{llrrrrr}
combustion & context & TC t$^{\text{-1}}$ \ac{PM2.5} & TC$_{\text{Cv}}$ t$^{\text{-1}}$ \ac{PM2.5} & EC t$^{\text{-1}}$ TC & EC$_{\text{Cv}}$ t$^{\text{-1}}$ \ac{PM2.5} & OC t$^{\text{-1}}$ TC\\
\hline
f & p & 0.621 & 0.07 & 0.023 & 0.15 & 0.598\\
f & wf & 0.608 & 0.09 & 0.1108 & 0.506 & 0.4976\\
s & p & 0.587 & 0.03 & 0.02 & 0.41 & 0.5675\\
s & wf & 0.641 & 0.08 & 0.045 & 0.29 & 0.59625\\
\end{tabular}
\end{table}

Given the variance in \ac{BC} production from smoldering (\textpm{} 49\%) and flaming (\textpm{} 45\%) phases (Table \ref{tab:bc_pm}), actual emissions of \ac{BC}  may vary substantially depending on combustion. In addtion to these estimates \cite{Chow2010} provides an alternative source for estimates of \ac{BC} and \ac{OC} emissions in the state in 2006. Further work is necessary to evaluate the impacts of \ac{OC} on the net \ac{CO2e} emissions from pile burning. \cite{Pokhrel2016} estimated the absorptive properties of \ac{OC} to be 1.5 - 2.5 that of \ac{BC}. \cite{Chow2010} estimated that on 2006 29,530 Mt of \ac{OC} was emitted from wildfires. 
% Emacs 24.4.1 (Org mode 8.2.10)
\end{document}
